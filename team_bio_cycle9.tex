\documentclass[11pt]{amsart}
\usepackage[margin=0.75in]{geometry}                % See geometry.pdf to learn the layout options. There are lots.
\geometry{letterpaper}                   % ... or a4paper or a5paper or ... 
%\geometry{landscape}                % Activate for for rotated page geometry
%\usepackage[parfill]{parskip}    % Activate to begin paragraphs with an empty line rather than an indent
\usepackage{graphicx}
\usepackage{amssymb}
\usepackage{epstopdf}

\DeclareGraphicsRule{.tif}{png}{.png}{`convert #1 `dirname #1`/`basename #1 .tif`.png}


%\date{}                                           % Activate to display a given date or no date

\begin{document}


\section{Principal Investigator and Co-Investigator Biographical and
Publication Data}

Dr. Peregrine McGehee is an adjunct professor of Astronomy at College of the Canyons in Santa Clarita, California. His PhD is from New Mexico State University in 2005. He is active
in the fields of star formation and the interstellar medium, using observations from diverse facilities including the Sloan Digital Sky Survey, CARMA, 
the Caltech Submillimeter Observatory, and Herschel.
From 2007 to 2015 he was a major participant in the Planck mission data analysis including production of the Early Release Compact Source Catalog. 
He also served as the NASA P.I. on the Herschel key programme on Galactic Cold Cores.   
Peregrine has led the Thirty Meter Telescope Key Program proposal entitled {\it Star formation processes on AU scales} as an active member of the Star and Planet Formation International Science Definition Team and also contributed to WISE star formation studies as a member of the Galactic Science working group.
He is also a co-chair of the LSST Stars, Milky Way, and Local Volume Science Collaboration.

\subsection{Relevant recent team member publications} 

\begin{enumerate}
\item Magnetic fields in the infrared dark cloud G34.43+0.24,
Soam, A. et al. 2019, ApJ in press, arXiv:1908.03624
\item Synthetic observations of dust emission and polarisation of Galactic cold clumps, 
Juvela M. et al.,  2019, A\&A, in press, arXiv:1908.03421
\item Statistical analysis of the interplay between magnetic fields and filaments hosting Planck Galactic Cold Clumps, 
Alina, D. et al.  2019,  MNRAS, 485, 2825
\item Dust spectrum and polarisation at 850 $\mu$m in the massive IRDC G035.39-00.33,
Juvela, M. et al.,  2018, A\&A,  620A, 26
\item A Holistic Perspective on the Dynamics of G035.39-00.33: 
The Interplay between Gas and Magnetic 
Fields,
Liu T. et al., 2018, ApJ 859, 151 
\item The CARMA-NRO Survey,
Kong, S. et al., 2018, ApJS, 236, 25
\item The TOP-SCOPE Survey of Planck Galactic Cold Clumps: Survey Overview and Results of an Exemplar Source, PGCC G26.53+0.17,
Liu, T.  et al. , 2018, ApJS, 234, 28
\item Polarization measurement analysis. III. Analysis of the polarization angle dispersion function with high precision polarization data.
Alina, D. et al. 2016, A\&A, 595, A57 
\item Galactic cold cores. VI. Dust opacity spectral index,
Juvela, M.  et al.  2015, A\&A 584, A94
\item Galactic Cold Cores.V. Dust opacity, 
Juvela, M. et al., 2015, A\&A, 584, 93 
\item Galactic Cold Cores. IV. Cold submillimetre sources: catalogue and statistical analysis, 
Montillaud, J. et al., 2015, A\&A, 584, 92 
\item Polarization measurement analysis. II. Best estimators of polarization fraction and angle,
Montier, L. et al , 2015, A\&A, 574, A136 
\item Polarization measurement analysis. I. Impact of the full covariance matrix on polarization fraction and angle measurements, 
Montier, L. et al. 2015,  A\&A, 574, A135 
\item Dust properties inside molecular clouds from coreshine modeling and observations, 
Lefevre, C et al., 2014, A\&A, 572, 20 
\item Multiwavelength study of the high-latitude cloud L1642: chain of star formation, 
Malinen, J. et al., 2014, A\&A, 563, 125 
\item Galactic cold cores. III. General cloud properties,
Juvela, M. et al, 2012, A\&A, 541, 12 
\item Galactic cold cores: II. Herschel study of the extended dust emission around the first Planck detections, 
Juvela, M. et al., 2011, A\&A, 527, 111 
\item Galactic cold cores: Herschel study of first Planck detections, 
Juvela, M. et al., 2010, A\&A, 518, 93  
\end{enumerate}

\newpage

Dr. Mika Juvela works as university lecturer in the University of Helsinki, Finland. He has wide experience of ISM observations from near-infrared to radio wavelengths and is an expert in the modelling of line and dust continuum data (including dust polarisation). Dr. Juvela is the co-coordinator of the Planck project on cold cores and the PI of the Herschel key programme Galactic Cold Cores. Dr. Juvela will participate in the analysis and radiative transfer modelling of the observations. He has experience of the analysis of polarisation measurements from Planck and JCMT/POL-2.

\vspace{5mm}
Dr. Tie Liu is an EACOA fellow at East Asian observatory. He was granted his PhD in 2013 from Peking University, after which he
was a visiting post-doc at the Peking University and University of Chile and then a KASI fellow 
at the Korea Astronomy and Space science Institute (KASI)
His role includes joint polarization observations with JCMT/POL-2 and molecular line observations with JCMT, TRAO, and  KVN telescopes.

\vspace{5mm}
Dr Isabelle Ristorcelli is a CNRS scientist at IRAP, Toulouse, France. Her main research interests are the studies of the earliest phases of star formation 
in the Galaxy and of the properties and evolution of dust from diffuse medium to molecular clouds. She has experience in the observations with 
space and ground-based telescopes from the infrared to mm wavelength range. These last years she has been particularly active in the analysis 
of polarization data working on Planck and the PILOT balloon-born experiment. She has been the co-PI of the Planck-Herschel program 'Galactic Cold Cores'.

As part of this project, she will work on the analysis of the variation of the polarization fraction with density and with the angular dispersion function; 
the results will be compared with the properties derived from Planck polarization measurements at larger scale.

\vspace{5mm}

Dr. Archana Soam is a post-doctoral fellow at the SOFIA Science Center, working with Dr. Andersson on the origins and uses of dust-induced polarimetry. She has experience in observational astronomy using dust polarization data from the optical to sub-mm wavelengths. She is an active member of the B-Fields In STar-forming RegiOns (BISTRO) survey at the JCMT. 

\vspace{5mm}

Dr. Julien Montilllaud is an associate professor at the University of Bourgogne Franche-Comt\'e. 
He is experienced in the analysis of mid- and far-IR dust emission, and of millimeter molecular lines in the context of star formation.


\vspace{5mm}

Dr Kate Pattle is a postdoctoral researcher working with Prof. Walter Gear at the National University of Ireland Galway.  She has experience of working with a range of submillimetre instruments, including the JCMT and Herschel.  She is a highly experienced user of the JCMT POL-2 polarimeter, leading the JCMT BISTRO Survey data reduction team and working as a member of the POL-2 commissioning team.

\vspace{5mm}

Dr. Veli-Matti Pelkonen gained his PhD from the University of Helsinki in Nov 2009. During his post-doctoral career, including 2009 - 2011 in IPAC, he has worked with Herschel and Planck satellites' data, including being the person responsible for data reduction of the Herschel Open Time Key Programme "Galactic Cold Cores". He has two first-author papers on the modelling of polarized thermal dust emission, as well as contributions to other papers studying the role of magnetic fields in interstellar clouds. He will contribute to the data analysis, in particular the comparison to the models of grain alignment.

\vspace{5mm}
Mika Saajasto is a Ph.D. student working under the supervision of Mika Juvela. Saajasto has strong expertise in observational astronomy ranging from visual to radio wavelengths. His latest work has been concentrated on working with observations from Herschel space observatory and with several follow up observation projects related to molecular line observations.



\end{document}  